\section{Levi-Civita}
We can represent the cross product using {\bf Levi-Civita Symbol}

\begin{definition}[Levi-Civita]
	Let $a, b, c \in {1,2,3}$. Then we write:
	\begin{equation}
		\label{eq:levi-civita}
		\varepsilon_{abc}
		=
		\begin{cases}
			0  \ \ \ \ \  \text{if } \space a = b = c  \space \space \text{ or more generally} \space a,b,c \space \text{is {\bf not} permutation of } \space 1,2,3 \\
			+1  \ \ \ \ \  \text{if } \space a,b,c \space \text{ is an {\bf even} permuation of } \space 1,2,3                                                      \\
			-1  \ \ \ \ \ \text{if } \space a,b,c \space \text{ is an {\bf odd} permuation of } \space 1,2,3
		\end{cases}
	\end{equation}
\end{definition}

\begin{note}
	Value of $\varepsilon_{abc}$ depends on the {\bf parity} of the permutatiom.
\end{note}

\subsection{Cross product of Orthonoarmal Basis Using Levi-Civita}
\begin{definition}
	Then we can write the cross product {\bf Orthonoarmal Basis Vectors} $\underline{e_1}, \underline{e_2}, \underline{e_3}$ in the following way
	$$\underline{e_a} \times \underline{e_b} = \sum_{i=1}^{3}a_{k}b_{k}\varepsilon_{abc}$$

\end{definition}
