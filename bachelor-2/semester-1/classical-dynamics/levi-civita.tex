\section{Levi-Civita}
We can represent the cross product using {\bf Levi-Civita Symbol}

\begin{definition}[Levi-Civita]
	Let $a, b, c \in {1,2,3}$. Then we write:
	\begin{equation}
		\label{eq:levi-civita}
		\varepsilon_{abc}
		=
		\begin{cases}
			0  \ \ \ \ \  \text{if } \space a = b = c  \space \space \text{ or more generally} \space a,b,c \space \text{is {\bf not} permutation of } \space 1,2,3 \\
			+1  \ \ \ \ \  \text{if } \space a,b,c \space \text{ is an {\bf even} permuation of } \space 1,2,3                                                      \\
			-1  \ \ \ \ \ \text{if } \space a,b,c \space \text{ is an {\bf odd} permuation of } \space 1,2,3
		\end{cases}
	\end{equation}
\end{definition}

\begin{note}
	Value of $\varepsilon_{abc}$ depends on the {\bf parity} of the permutatiom.
\end{note}

\subsection{Cross product of Orthonoarmal Basis Using Levi-Civita}
\begin{definition}
	Then we can write the cross product {\bf Orthonoarmal Basis Vectors} $\underline{e_1}, \underline{e_2}, \underline{e_3}$ in the following way
	\begin{equation}
		\label{eq: levi-orthonormal}
		\underline{e_a} \times \underline{e_b} = \sum_{i=1}^{3}a_{k}b_{k}\varepsilon_{abc}
	\end{equation}

\end{definition}

\subsection{Cross Product of Vectors in Levi-Civita Notation}
\begin{theorem}[Cross Product using Levi-Civita Notation]
	Let $\vec{a}, \vec{b} \in \mathbb{E}^3$. Then
	\begin{equation}
		\label{eq:levi-civita-cross-product}
		\vec{a} \times \vec{b} = \sum_{m=1}^{3}(\vec{a} \times \vec{b})_{m}\underline{e_m}
	\end{equation}
	where $(\vec{a} \times \vec{b})_m$ is the {\bf mth component}
	\begin{equation}
		\label{eq: mth-component}
		(\vec{a} \times \vec{b})_m = \sum_{k, \ l}\varepsilon_{abc}a_{k}b_{l}
	\end{equation}
\end{theorem}
\begin{proof}
	Let
	$$
		\underline{a} = a_{k}\underline{e_{k}} = \sum_{k=1}^{3}a_{k}\ \underline{e_{k}} \hspace{100px} \underline{b} = b_{l}\underline{e_{l}}= \sum\limits_{l=1}^{3} b_{l}\ \underline{e_{l}}
	$$
	Observe the use of {\bf Eintein's Notation} (see below) and observe that
	\begin{align*}
		\underline{a} \times \underline{b} & = \Big( \sum\limits_{k = 1}^{3}a_k\underline{e_{k}} \Big)\ \times \ \Big( \sum\limits_{l = 1}^{3}b_l\underline{e_{l}} \Big) \\ \\
		                                   & = \sum\limits_{k, \ l}a_{k} \ b_{k} \ \underline{e_{k}} \times \underline{e_{l}}
	\end{align*}
	And therefore by \ref{eq: levi-orthonormal}, we can rewrite it in the following way
	$$\hspace{-80px} = \sum\limits_{k,\ l,\ m} a_{k} \ b_{l} \ \epsilon_{klm} \ \underline{e_m}$$
	Define the {\bf mth component} as
	$$(\vec{a} \times \vec{b})_m =  \sum_{k, \ l}\varepsilon_{abc}a_{k}b_{l}$$
	and hence we get \ref{eq:levi-civita-cross-product}
\end{proof}
