\subsection{N-body Gravitational System}
For $N$-body Gravitational System with no external forces, moving under mutual {\bf gravitational forces}, we calculate the {\bf rate of change of Kinetic Energy} \eqref{eq: kinetic-energy} of the system.

\begin{align*}
	\dot{K}_{tot} & =\frac{d}{dt}\sum_{i=1}^{N}\frac{1}{2}m_i \mid \underline{\dot{r}}_i \mid^{2} \\ \\
	              & = \sum_{i=1}^{N}m_{i}\underline{\dot{r}}_i \cdot \underline{\ddot{r}}_i
\end{align*}

\begin{note}
	For a gravitational system
	$$m_i \underline{\ddot{r}}_i = \frac{Gm_{i}m_{j}}{\mid \underline{r_i} - \underline{r_j} \mid^{2}} \frac{\underline{r_i} - \underline{r_j}}{\mid \underline{r_i} - \underline{r_j} \mid }$$
\end{note}

We can write the {\bf rate of change of Kinetic Energy} \eqref{eq: kinetic-energy} of the system as

\begin{align*}
	\dot{K}_{tot} & = \sum_{i=1}^{N} \underline{\dot{r}}_i  \cdot \sum_{j=1}^{N} \frac{Gm_{i}m_{j}}{\mid \underline{r_i} - \underline{r_j} \mid^{2}} \frac{\underline{r_i} - \underline{r_j}}{\mid \underline{r_i} - \underline{r_j} \mid }       \\ \\
	              & = \sum_{\stackrel{i=1}{i \neq j}} \frac{Gm_{i}m_{j}}{\mid \underline{r_i} - \underline{r_j} \mid^{3}} \ \ \underline{r_i} \cdot (\underline{r_i} - \underline{r_j})                                                           \\ \\
	              & = \frac{1}{2}\sum_{\stackrel{i=1}{i \neq j}}\frac{Gm_{i}m_{j}}{\mid \underline{r_i} - \underline{r_j} \mid^{3}} (\underline{r_i} - \underline{r_j}) \cdot (\underline{r_i} - \underline{r_j}) \tag{$*$} \label{ct: n-gravity}
\end{align*}

\begin{remark}
	$$\frac{d}{dt}\mid \underline{p}\mid^{2} = 2 \mid \underline{p} \mid \frac{d \mid\underline{p} \mid }{dt}$$
	and
	$$\frac{d}{dt}\mid \underline{p}\mid^{2} = \frac{d}{dt}(\underline{p} \cdot \underline{p}) = 2\underline{{p}}\cdot\underline{\dot{p}}  $$
	And therefore we get
	\begin{align*}
		2\mid \underline{p} \mid \frac{d\mid \underline{p}\mid }{dt} = 2\underline{p}\cdot\underline{\dot{p}} \\ \\
		\Rightarrow \frac{d\mid \underline{p}\mid }{dt} =  \underline{p}\cdot\underline{\dot{p}}
	\end{align*}
\end{remark}

\begin{note}
	\begin{align*}
		\frac{d}{dt}\frac{1}{\mid \underline{r_i} - \underline{r_j} \mid} & = - \frac{1}{\mid \underline{r_i} - \underline{r_j}\mid^{2}}\frac{d}{dt}\mid \underline{r_i} - \underline{r_j} \mid                       \\ \\
		                                                                  & = - \frac{1}{\mid \underline{r_i} - \underline{r_j} \mid^3 }(\underline{r_i} - \underline{r_j}) \cdot (\underline{r_i} - \underline{r_j})
	\end{align*}
\end{note}

Therefore equation \eqref{ct: n-gravity} becomes
\begin{align*}
  \dot{K}_{tot} & = \frac{1}{2}\sum_{\stackrel{i=1}{i \neq j}}^{N} \frac{d}{dt}\frac{Gm_i m_j}{\mid \underline{r_i} - \underline{r_j} \mid}      \\ \\
	        & = \sum_{i < j}^{N} \frac{d}{dt}\frac{Gm_i m_j}{\mid \underline{r_i} - \underline{r_j} \mid} \tag{$**$} \label{ct: n-gravity-2}
\end{align*}
Since $i<j$ is already {\bf half the number} of terms in the sum. \clearpage And using the {\bf summation properties for derivatives} \eqref{ct: n-gravity-2} becomes

\begin{align*}
  \dot{K}_{tot} & = \sum_{i < j}^{N} \frac{d}{dt}\frac{Gm_i m_j}{\mid \underline{r_i} - \underline{r_j} \mid} \\ \\
  &= \frac{d}{dt}\sum_{i < j}^{N} \frac{Gm_i m_j}{\mid \underline{r_i} - \underline{r_j} \mid} \\ \\
\end{align*}

Therefore we get
\begin{align*}
  \dot{K}_{tot} - \frac{d}{dt}\sum_{i < j}^{N} \frac{Gm_i m_j}{\mid \underline{r_i} - \underline{r_j} \mid}  = 0 & \Rightarrow \frac{dK_{tot}}{dt} -  \frac{d}{dt}\sum_{i < j}^{N} \frac{Gm_i m_j}{\mid \underline{r_i} - \underline{r_j} \mid} = 0 \\ \\
                                                                                                        & \Rightarrow \frac{d}{dt}\left( K_{tot} - \sum_{i < j}^{N} \frac{Gm_i m_j}{\mid \underline{r_i} - \underline{r_j} \mid}  \right) = 0 
\end{align*}
That is a {\bf Total Energy is conserved}

\begin{definition}[Total Energy in N-body system]
  The {total energy $E$ in an N-body gravity} system is
\begin{equation}
  E = K_{tot} - \sum_{i < j}^{N} \frac{Gm_i m_j}{\mid \underline{r_i} - \underline{r_j} \mid} 
  \label{eq: total-energy-n-body-grav}
\end{equation}
  or using \eqref{eq:total-kinetic-energy-2}, we get
  \begin{equation}
    E = \frac{1}{2}M\mid\underline{\dot{R}} \mid^{2} + \frac{1}{2}\sum_{i=1}^{N}m_i \mid \underline{\dot{s}_i}\mid^{2}  - \sum_{i < j}^{N} \frac{Gm_i m_j}{\mid \underline{r_i} - \underline{r_j} \mid}  
    \label{eq:total-energy-n-body-grav-2}
  \end{equation}
\end{definition}

\begin{definition}[Potential Energy in N-body gravitational system]
  The term
  \begin{equation}
     \sum_{i < j}^{N} \frac{Gm_i m_j}{\mid \underline{r_i} - \underline{r_j} \mid} 
    \label{eq:gravitational-potential-n-body}
  \end{equation}
  is the {\bf total gravitational potential energy} expressed as a {\em sum over all pairs of particles}.
\end{definition}
\clearpage
