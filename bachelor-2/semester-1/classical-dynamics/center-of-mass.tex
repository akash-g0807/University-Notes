\subsection{Center of Mass}
\begin{definition}[Center of Mass]
	In a discrete system of $N$ particles with {\bf masses} $m_i$ and {\bf position vectors} $\underline{r}_i$, relative to a fixed origin $O$, the {\bf center of mass} is defined as
	\begin{equation}
		\label{eq:com}
		\underline{R} = \frac{\sum\limits_{i=1}^{N}m_{i}\underline{r_{i}}}{\sum\limits_{i=1}^{N}m_{i}}
	\end{equation}
\end{definition}

\begin{note}
	The denominator of \eqref{eq:com} is the {\bf total mass} of the system which will be denoted by
	\begin{equation}
		\label{eq:total-mass}
		M = \sum\limits_{i=1}^{N}m_{i}
	\end{equation}
	and hence
	\begin{equation}
		\label{eq:com2}
		\underline{R} = \frac{\sum\limits_{i=1}^{N}m_{i}\underline{r_{i}}}{M}
	\end{equation}
\end{note}

\begin{definition}[Total External Force using Center of Mass]
	\label{def:total-external-force}
	The {\bf total external force} acting on the system is defined as
	\begin{equation}
		\label{eq:total-external-force-com}
		M\underline{\ddot{R}} = \underline{F^{(e)}_\text{total}}
	\end{equation}
\end{definition}

\begin{note}
	If {\bf total external force is zero}, i.e. $\underline{F^{(e)}_\text{total}} = 0$, then $\underline{\ddot{R}} = 0$ and therefore $\underline{\dot{R}}$ is contant.
	Hence the center of mass {\bf moves with constant velocity}.
\end{note}
