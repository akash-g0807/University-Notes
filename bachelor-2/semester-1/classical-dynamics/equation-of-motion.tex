\section{Equation of Motion}
\begin{note}
	Acceleration is {\bf proportional} to the {\bf net force} acting on the body. Therefore, we can write
	$$\underline{a} \propto \underline{F}$$

	In an {\bf inertial frame}, a particle moves  in such a way that its acceleration (\ref{eq: acceleration-particle}) is {\bf proportional} to the sum of all forces acting on it {\bf Newton's Second Law of Motion}
\end{note}

\begin{definition}[Equation of Motion]
	The {\bf equation of motion} of a particle is the {\bf differential equation} that describes the {\bf trajectory} of the particle in space. In an {\bf inertial frame}, the equation of motion is given by
	\begin{equation}
		\label{eq: equation-of-motion}
		\underline{\ddot{r}}(t) = \frac{\underline{F}}{m}
	\end{equation}
	where $\underline{F}$ is the {\bf net force} acting on the particle and $m$ is the {\bf mass} of the particle.
\end{definition}

\clearpage
\subsection{Momentum}
\begin{definition}[Momentum]
	The {\bf momentum} of a particle is the {\bf product} of its {\bf mass} and {\bf velocity}:
	\begin{equation}
		\label{eq: momentum}
		\underline{p} = m\underline{v}
	\end{equation}
\end{definition}

\begin{note}
	From (\ref{eq: momentum}), we can see that the {\bf momentum} is a {\bf vector} quantity.
\end{note}

We can generalize the definition of Force usng momentum as follows:
\begin{definition}[Newton's Second Law in terms of Momentum]
	Newton's second law (\ref{eq: newtons-second-law}) can be written in terms of momentum as follows:
	\begin{equation}
		\label{eq: newtons-second-law-momentum}
		\underline{F} = \frac{d\underline{p}}{dt}
	\end{equation}
\end{definition}
