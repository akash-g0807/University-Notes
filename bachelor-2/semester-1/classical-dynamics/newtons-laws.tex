\section{Newton's Laws of Motion}

\subsection{Inertia}
\begin{definition}[Law of Inertia]
	Any body which {\bf isn’t} being acted on by an {\bf outside force} stays at rest if it is {\em initially} at rest, or continues to move at a {\bf constant} velocity if that’s what it was doing to begin with. i.e.

	Every object will \emph{remain} at {\bf rest} or in {\bf uniform motion} in a {\bf straight line} unless {\em compelled} to change its state by the action of an {\bf external force}.

\end{definition}

\subsection{Newton's First Law of Motion}
\begin{definition}[Newton's First Law]
	Every body {\bf continues} in a state of {\bf rest} or {\bf uniform motion} in a right line unless t is {\em compelled} to change that state by {\bf forces} impressed on it.
\end{definition}

\subsection{Newton's Second Law of Motion}
\begin{definition}[Newton's Second Law]
	The {\em change} of motion is {\bf proportional} to the {\bf motive force} impressed on it and is *made* in the {\bf direction of the right line} in which that force was impressed.
\end{definition}

Newton's second law postulates a {\em relation} between acceleration (\ref{eq: acceleration-particle}) of the body and the {\bf forces} acting on it.  Therefore we can reformulate Newon's second law as follows:

\begin{definition}[Newton's Second Law]
	The {\bf net force} $\underline{F}$ on a body of {\bf constant mass} causes a body to {\bf accelerate}. The acceleration $\underline{\ddot{r}}$ is {\em in the direction} of $\underline{F}$ {\bf proportional} to the magnitude of the force and {\bf inversely proportional}  to the mass of the body:
	$$\ddot{\underline{x}} = \frac{\underline{F}}{m}$$
	or equivalently
	\begin{equation}
		\label{eq: newtons-second-law}
		\underline{F} = m\underline{\ddot{x}}
	\end{equation}

\end{definition}

\subsection{Newton's Third Law of Motion}
\begin{definition}[Newton's Third Law]
	To every {\bf action} there is always an {\bf equal and opposite reaction}: or the {\bf mutual actions} of two bodies upon each other are always {\bf equal} and {\bf directed} to {\bf contrary} parts.
\end{definition}
