\lesson{1}{2023-10-23 16:36}{Newtownian Dynamics}
\chapter{Newtownian Dynamics}

In this section, we deal with {\bf particles} Particles are an idealization since real objects even if very small have very small a spatial extent.

	{\bf Particles} will be represented by a {\bf point in space} that movies in a trajectory denoted by $\underline{r}(t)$ which is a vector denoting its position at a time $t$ {\em relative} to a specified {\bf origin}.

\section{Basic Kinematics}
\subsection{Position of a particle}
\begin{definition}
	A point {\bf particle's} {\bf positon} at time $t$ on a {\bf trajectory} relative to an {\bf origin} $O$ can be described by a {\bf position vector}
		{\em relative} to an origin $O$

	The {\bf position vector} is $\underline{r}$ and can be represented using {\bf basis vectors}.
	\begin{equation}
		\label{eq: position-vector-particle}
		\underline{r}(t) = x\underline{i} + y\underline{j} + z\underline{k}
	\end{equation}

\end{definition}
\vspace{-10px}
\begin{mycenter}
	\tikzset{every picture/.style={line width=0.75pt}} %set default line width to 0.75pt        

	\begin{tikzpicture}[x=0.75pt,y=0.75pt,yscale=-1,xscale=1]
		%uncomment if require: \path (0,402); %set diagram left start at 0, and has height of 402

		%Curve Lines [id:da7249822784379534] 
		\draw    (108.1,195.3) .. controls (184.1,20.3) and (248.1,240.3) .. (367.1,64.3) ;
		%Straight Lines [id:da07486429812279405] 
		\draw    (203.5,258.3) -- (244.75,137.59) ;
		\draw [shift={(245.4,135.7)}, rotate = 108.87] [color={rgb, 255:red, 0; green, 0; blue, 0 }  ][line width=0.75]    (10.93,-3.29) .. controls (6.95,-1.4) and (3.31,-0.3) .. (0,0) .. controls (3.31,0.3) and (6.95,1.4) .. (10.93,3.29)   ;
		%Shape: Circle [id:dp6919219182514531] 
		\draw  [fill={rgb, 255:red, 0; green, 0; blue, 0 }  ,fill opacity=1 ] (242,135.7) .. controls (242,133.82) and (243.52,132.3) .. (245.4,132.3) .. controls (247.28,132.3) and (248.8,133.82) .. (248.8,135.7) .. controls (248.8,137.58) and (247.28,139.1) .. (245.4,139.1) .. controls (243.52,139.1) and (242,137.58) .. (242,135.7) -- cycle ;
		%Shape: Circle [id:dp2798104629473662] 
		\draw  [fill={rgb, 255:red, 0; green, 0; blue, 0 }  ,fill opacity=1 ] (200.1,258.3) .. controls (200.1,256.42) and (201.62,254.9) .. (203.5,254.9) .. controls (205.38,254.9) and (206.9,256.42) .. (206.9,258.3) .. controls (206.9,260.18) and (205.38,261.7) .. (203.5,261.7) .. controls (201.62,261.7) and (200.1,260.18) .. (200.1,258.3) -- cycle ;

		% Text Node
		\draw (213,245) node [anchor=north west][inner sep=0.75pt]   [align=left] {$\displaystyle O$};
		% Text Node
		\draw (241,108) node [anchor=north west][inner sep=0.75pt]   [align=left] {$\displaystyle P$};
		% Text Node
		\draw (236,173) node [anchor=north west][inner sep=0.75pt]   [align=left] {$\displaystyle \underline{r}( t)$};


	\end{tikzpicture}
\end{mycenter}

\begin{note}[Using Einstein Notation to describe position]

	Using {\bf Einstein's Notation} we can also represented it in the following way:
	$$\underline{r}(t) = \lambda_{a}\underline{e_{a}}$$

\end{note}

\subsection{Kinematics: Velocity and Acceleration}
In position vector of a particle (\ref{eq: position-vector-particle})  $\underline{r}$ assuming that the {\bf Orthonormal bais unit vectors} $\underline{i}, \underline{j}$ and $\underline{k}$ are {\bf constant}, we can write {\bf velocity} and {\bf acceleration} in the following way:

\clearpage

\subsubsection{Velocity}
Consider the following diagram:
\begin{mycenter}
	\tikzset{every picture/.style={line width=0.75pt}} %set default line width to 0.75pt        

	\begin{tikzpicture}[x=0.75pt,y=0.75pt,yscale=-1,xscale=1]
		%uncomment if require: \path (0,402); %set diagram left start at 0, and has height of 402

		%Curve Lines [id:da7249822784379534] 
		\draw    (108.1,195.3) .. controls (184.1,20.3) and (284.1,243.3) .. (403.1,67.3) ;
		%Straight Lines [id:da07486429812279405] 
		\draw    (203.5,258.3) -- (205.37,127.7) ;
		\draw [shift={(205.4,125.7)}, rotate = 90.82] [color={rgb, 255:red, 0; green, 0; blue, 0 }  ][line width=0.75]    (10.93,-3.29) .. controls (6.95,-1.4) and (3.31,-0.3) .. (0,0) .. controls (3.31,0.3) and (6.95,1.4) .. (10.93,3.29)   ;
		%Shape: Circle [id:dp6919219182514531] 
		\draw  [fill={rgb, 255:red, 0; green, 0; blue, 0 }  ,fill opacity=1 ] (202,125.7) .. controls (202,123.82) and (203.52,122.3) .. (205.4,122.3) .. controls (207.28,122.3) and (208.8,123.82) .. (208.8,125.7) .. controls (208.8,127.58) and (207.28,129.1) .. (205.4,129.1) .. controls (203.52,129.1) and (202,127.58) .. (202,125.7) -- cycle ;
		%Shape: Circle [id:dp2798104629473662] 
		\draw  [fill={rgb, 255:red, 0; green, 0; blue, 0 }  ,fill opacity=1 ] (200.1,258.3) .. controls (200.1,256.42) and (201.62,254.9) .. (203.5,254.9) .. controls (205.38,254.9) and (206.9,256.42) .. (206.9,258.3) .. controls (206.9,260.18) and (205.38,261.7) .. (203.5,261.7) .. controls (201.62,261.7) and (200.1,260.18) .. (200.1,258.3) -- cycle ;
		%Shape: Circle [id:dp4797202017377725] 
		\draw  [fill={rgb, 255:red, 0; green, 0; blue, 0 }  ,fill opacity=1 ] (244,132.7) .. controls (244,130.82) and (245.52,129.3) .. (247.4,129.3) .. controls (249.28,129.3) and (250.8,130.82) .. (250.8,132.7) .. controls (250.8,134.58) and (249.28,136.1) .. (247.4,136.1) .. controls (245.52,136.1) and (244,134.58) .. (244,132.7) -- cycle ;
		%Straight Lines [id:da797119100975235] 
		\draw [color={rgb, 255:red, 208; green, 2; blue, 27 }  ,draw opacity=1 ]   (203.5,258.3) -- (246.74,134.59) ;
		\draw [shift={(247.4,132.7)}, rotate = 109.27] [color={rgb, 255:red, 208; green, 2; blue, 27 }  ,draw opacity=1 ][line width=0.75]    (10.93,-3.29) .. controls (6.95,-1.4) and (3.31,-0.3) .. (0,0) .. controls (3.31,0.3) and (6.95,1.4) .. (10.93,3.29)   ;
		%Straight Lines [id:da3151106945356048] 
		\draw [color={rgb, 255:red, 126; green, 211; blue, 33 }  ,draw opacity=1 ]   (205.4,125.7) -- (245.43,132.37) ;
		\draw [shift={(247.4,132.7)}, rotate = 189.46] [color={rgb, 255:red, 126; green, 211; blue, 33 }  ,draw opacity=1 ][line width=0.75]    (10.93,-3.29) .. controls (6.95,-1.4) and (3.31,-0.3) .. (0,0) .. controls (3.31,0.3) and (6.95,1.4) .. (10.93,3.29)   ;

		% Text Node
		\draw (213,245) node [anchor=north west][inner sep=0.75pt]   [align=left] {$\displaystyle O$};
		% Text Node
		\draw (201,96) node [anchor=north west][inner sep=0.75pt]   [align=left] {$\displaystyle P$};
		% Text Node
		\draw (172,177) node [anchor=north west][inner sep=0.75pt]   [align=left] {$\displaystyle \underline{r}( t)$};
		% Text Node
		\draw (229,183) node [anchor=north west][inner sep=0.75pt]  [color={rgb, 255:red, 208; green, 2; blue, 27 }  ,opacity=1 ] [align=left] {$\displaystyle \underline{r}( t\ +\ \delta t)$};
		% Text Node
		\draw (221,105) node [anchor=north west][inner sep=0.75pt]  [color={rgb, 255:red, 126; green, 211; blue, 33 }  ,opacity=1 ] [align=left] {$\displaystyle \delta \underline{r}$};


	\end{tikzpicture}

\end{mycenter}
From the diagram above:
$$\delta\underline{r} = \underline{r}(t+\delta t) - \underline{r}(t) $$
Dividing by $\delta t$ and taking the {\bf limit} as $\delta \rightarrow 0$ we get
$$\dot{\underline{r}}(t) = \underline{v}(t) = \lim_{\delta t \to 0}\bigg(\frac{\underline{r}(t + \delta t) - \underline{r}(t)}{\delta t}\bigg)$$

\begin{definition}[Velocity of a Particle]
	\begin{equation}
		\label{eq: velocity-particle}
		\dot{\underline{r}} = \frac{d\underline{r}(t)}{dt}  = \lambda \dot{\underline{e_{a}}} = \dot{x}\underline{i} + \dot{y}\underline{j} +\dot{z}\underline{k}
	\end{equation}
\end{definition}

\subsubsection{acceleration}
Similarly {\bf acceleration} can be definined in the following way:
\begin{definition}[Velocity of a Particle]
	\begin{equation}
		\label{eq: acceleration-particle}
		\ddot{\underline{r}} = \frac{d\dot{\underline{r}}(t)}{dt}  = \lambda \ddot{\underline{e_{a}}} = \ddot{x}\underline{i} + \ddot{y}\underline{j} +\ddot{z}\underline{k}
	\end{equation}
\end{definition}

In terms of {\bf limits}
$$\ddot{\underline{r}}(t) = \dot{\underline{v}}(t) = \underline{a}(t) = \lim_{\delta t \to 0}\bigg(\frac{\underline{v}(t + \delta t) - \underline{v}(t)}{\delta t}\bigg)$$


\clearpage
