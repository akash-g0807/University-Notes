\section{Angular Momentum}

\begin{definition}[Angular Momentum]
	The {\bf angular momentum} $\underline{J}$ of a particle is given by:
	\begin{equation}
		\label{eq: angular-momentum}
		\underline{J} = \underline{r} \times \underline{p} = m\underline{{r}} \times \underline{\dot{r}}
	\end{equation}
	where $\underline{r}$ is the position vector of the particle, and $\underline{p}$ is the momentum (\ref{eq: momentum}) of the particle.
\end{definition}

\clearpage
\subsection{Moment of a Force}
From the following calculation:

$$\begin{aligned} \underline{\dot{J}} & = \frac{d}{dt}\Big(m\underline{{r}} \times \underline{\dot{r}}\Big)                                                                      \\ \\ &= m\underline{\dot{r}} \times \underline{\dot{r}} + m\underline{{r}} \times \underline{\ddot{r}}\ \ \ \ \ \ \ \ \ \ \ \ \ \ \ \text{product rule} \\ \\
                                    & = 0 + m\underline{r}  \times \underline{\ddot{r}} \ \ \ \ \ \ \ \ \ \ \ \ \ \ \ \ \text{properties of cross product}                     \\ \\
                                    & = m\underline{r} \times \underline{\ddot{r}}                                                                                             \\\\
                                    & = \underline{r} \times \underline{F} \ \ \ \ \ \ \ \ \ \ \ \ \ \ \ \ \ \text{Newton's Equation of Motion (\ref{eq: equation-of-motion})} \\ \\
                                    & \equiv  \underline{M}
	\end{aligned}$$

\begin{definition}[Moment of a Force]

	The {\bf moment of a force} $\underline{M}$  of a particle is defined to the {\bf rate of change of angular momentum} (\ref{eq: angular-momentum}) is given by:
	\begin{equation}
		\label{eq: moment-of-a-force}
		\underline{M} = \underline{r} \times \underline{F}
	\end{equation}
	where $\underline{r}$ is the position vector of the particle, and $\underline{F}$ is the {\bf force} on the particle.
\end{definition}

\begin{note}
	$\underline{M}$ is also called the {\bf torque} of the force $\underline{F}$.
\end{note}

\subsection{Conservation of Angular Momentum}
Consider a particle moving under the influence of a force $\underline{F}$ directed towards or away from the origin.
$$\underline{F} = f(\underline{r})\underline{r} $$
where $f(\underline{r})$ is a scalar. Hence calculating the moment of the force $\underline{F}$:
$$\begin{aligned} \underline{\dot{J}} = \underline{r} \times \underline{F} & = f(\underline{r})\underline{r} \times \underline{r} \\ \\
                                                                         & = 0                                                  \\ \\
                \Rightarrow \underline{\dot{J}} = 0\end{aligned}$$
Hence angular momentum is a {\bf conserved quantity}.

\clearpage
\begin{definition}[Conservation of Angular Momentum]
	If a force $\underline{F}$ is {\bf proportional} to $\underline{r}$, i.e.
	$$\underline{F}=f(\underline{r})$$
	then angular momentum is conserved:
	$$\underline{\dot{J}} = 0$$

\end{definition}
