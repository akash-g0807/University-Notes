\subsection{A Virial Theorem}
Define the following:
$$D = \frac{1}{2}\sum_{i-1}^{N}m_{i} \mid \underline{r_{i}}^{2} \mid$$
Then we get the following for the derivatives
$$\dot{D} = \sum_{i=1}^{N}m_{i}\underline{r_{i}} \cdot \underline{\dot{r_{i}}} \ \ \ \ \ \ \text{chain
rule}$$
and the second derivative is (from the product rule)
\begin{align*}
  \ddot{D} = \sum_{i=1}^{n}m_{i}\underline{\dot{r_{i}}}\cdot \underline{\dot{r_{i}}} + \sum_{i=1}^{n}m_{i}\underline{\dot{r_{i}}}\cdot\underline{\ddot{r_{i}}}\label{eq: virial-second-derivative} \tag{$*$}
\end{align*}

Then therefore we can rewrite the equation \eqref{eq: virial-second-derivative}
using definition of \eqref{eq: kinetic-energy}
$$\ddot{D} = 2K_{tot} +   \sum_{i=1}^{n}m_{i}\underline{\dot{r_{i}}}\cdot\underline{\ddot{r_{i}}} $$
\subsubsection{Virial Theorem on Gravity}
{\bf Gravitational Force of Attraction} is defined as
$$m\underline{\ddot{r}_{i}} = \frac{Gm_{i}m_{j}}{\mid\underline{r_{i}} -\underline{r_{j}}\mid^{2}}
\frac{\underline{r_{j}} - \underline{r_{i}}}{\mid \underline{r_{i}} - \underline{r_{i}} \mid}$$
And therefore substituting this into the {\bf second derivative} $\ddot{D}$ we
get the following:

\begin{align*}
  \ddot{D} &= 2K_{tot} + \sum_{i=1}^{N}\underline{r_{i}} \cdot \sum_{i \neq j}\frac{Gm_{i}m_{j}}{\mid \underline{r_{i}}- \underline{r_{j}} \mid} \frac{\underline{r_{j}} - \underline{r_{i}}}{\mid \underline{r_{i}} - \underline{r_{j}} \mid} \\ \\
           &= 2K_{tot} + \frac{1}{2}\sum_{i \neq j}(\underline{r_{i}} - \underline{r_{j}}) \cdot \frac{Gm_{i}m_{j}}{\mid \underline{r_{i}} - \underline{r_{j}} \mid^{3}}(\underline{r_{i}} - \underline{r_{j}}) \\ \\
           &= 2K_{tot} + \Phi \\ \\
           &= K_{tot} + K_{tot} + \Phi \\ \\
  \Rightarrow \ddot{D} = K_{tot} + \underbrace{E}_{\text{conserved}}
\end{align*}
\clearpage

Define {\bf average Kinetic Energy}
\begin{definition}[Average Kinetic Energy]
  \begin{equation}
    \label{eq: average-kinetic-energy}
    <K_{tot}> = \frac{1}{\tau}\int_{0}^{\tau}K_{tot}dt
  \end{equation}
\end{definition}

Suppose the quantity $\underline{R}$ does not change, we find that
$$\underline{E} = -<K_{tot}> \ \ \ \text{or} \ \ \ 2 <K_{tot}> = -<V_{tot}>$$
This fact was the basis of an analysis of the Coma cluster of galaxies by Zwicky (‘On
the Masses of Nebulae and of Clusters of Nebulae’, F Zwicky, Astrophysical Journal, vol.
86 (1937) 217), which demonstrated that there should be some kind of ‘dark matter’ to
account for observation. So far, ‘dark matter’ has not been identified directly though there
are other, independent, indications that it should exist and many theories as to what it
might be. (For example, see ‘Particle dark matter: evidence, candidates and constraints’,
G Bertone, D Hooper, J Silk, Physics Reports 405 (2005) 279).
