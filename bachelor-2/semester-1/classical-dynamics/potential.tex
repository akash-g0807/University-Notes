\subsection{Potential Energy}
\begin{definition}[Conservative Forces]
	A force $\underline{F}$ is {\bf conservative} if it can be written as the {\bf gradient} of a {\bf scalar function} $\Phi$:
	$$\underline{F} = -\nabla \Phi$$
	where $\nabla$ is the {\bf gradient operator}:
	$$\nabla = \Big(\frac{\partial}{\partial x}, \frac{\partial}{\partial y}, \frac{\partial}{\partial z}\Big)$$
\end{definition}
Hence the work done by a conservative force $\underline{F}$ is given by:

$$\underline{F} = - \underline{\nabla}\Phi = - \Big(\frac{\partial}{\partial x} \underline{i} + \frac{\partial}{\partial y} \underline{j} + \frac{\partial}{\partial z} \underline{k}\Big)\Phi = - \Big(\frac{\partial \Phi}{\partial x} \underline{i} + \frac{\partial \Phi}{\partial y} \underline{j} + \frac{\partial \Phi}{\partial z} \underline{k}\Big)$$

\subsubsection{Potential Energy and Conservation}
Consider the following calculations:
$$\begin{aligned} \underline{F} = - \underline{\nabla}\Phi &\Rightarrow \underline{F} \cdot \underline{r} = - \underline{r} \cdot \underline{\nabla}\Phi \end{aligned}$$
Now by the definition of Kinetic Energy (\ref{eq: kinetic-energy})

$\underline{F} \cdot \underline{r} = dK/dt = \dot{K}$, we get the following:
$$\begin{aligned} \frac{dK}{dt} = -\underline{\dot{r}} \cdot \underline{\nabla}\Phi & \Rightarrow  \frac{dK}{dt} = -\underline{\dot{r}}\cdot\underline{\nabla}\Phi(\underline{r}) \\ \\
                                                                                  & \Rightarrow \frac{dK}{dt} = -\frac{d\Phi}{dt} \ \ \ \ \ \ \ \ \ \ \ \ \ \ \text{chain rule} \\ \\
                                                                                  & \Rightarrow \frac{d}{dt}\Big(K + \Phi\Big) = 0\end{aligned}$$
And therefore Energy is a {\bf conserved quantity}.
\clearpage
\begin{definition}[Conservation of Energy]
	Energy is a {\bf constant of motion}
	$$\dot{E} = \frac{d}{dt}\Big(K + \Phi\Big) = 0$$
	Therefore
	$$E = K + \Phi = \text{CONSTANT}$$
\end{definition}

\begin{definition}[Potential Energy]
	The {\bf potential energy} $\Phi$ is given by:
	\begin{equation}
		\label{eq: potential-energy}
		\Phi = -\int_{P_{1}}^{P_{2}}\underline{F} \cdot d\underline{r}
	\end{equation}
\end{definition}
