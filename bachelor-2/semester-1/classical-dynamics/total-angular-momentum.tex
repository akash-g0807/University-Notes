\subsection{Total Angular Momentum}

Consider the following calculations:
\begin{align*}
	\underline{J}_{tot} & = \sum_{i=1}^{N}m_{i} \underline{r_i} \times \underline{\dot{r_i}}                                                                                                                                  \\ \\
	                    & = \sum_{i=1}^{N}m_{i}(\underline{R} + \underline{s_i}) \times (\underline{\dot{R}} + \underline{\dot{s_i}}) \ \ \ \ \ \ \ \ \ \ \ \ \text{since } \underline{r_i} = \underline{s_i} + \underline{R} \\ \\
	                    & = M \underline{R} \times \underline{\dot{R}} + \frac{1}{2}\sum_{i=1}^{N}m_{i}\underline{s_{i}} \times \underline{\dot{s_{i}}}
\end{align*}

\begin{definition}[Total Angular Momentum]
	\label{def:total-angular-momentum}
	In a discrete system of $N$ particles with {\bf masses} $m_i$ and {\bf position vectors} $\underline{r}_i$, relative to a fixed origin $O$, the {\bf total angular momentum} of a collection of particles is defined as
	\begin{equation}
		\label{eq:total-angular-momentum}
		\underline{J}_{tot} = M \underline{R} \times \underline{\dot{R}} + \frac{1}{2}\sum_{i=1}^{N}m_{i}\underline{s_{i}} \times \underline{\dot{s_{i}}}
	\end{equation}
\end{definition}
