\section{Kronecker-Delta}

As shown before, the properties of the {\bf scalar product}, the {\bf orthonormal basis} vectors have the following properties
$$\underline{e_{1}} \cdot \underline{e_{2}}= 0 = \underline{e_{1}} \cdot \underline{e_{3}} = \underline{e_{2}} \cdot \underline{e_{3}}$$
and
$$\underline{e_{1}} \cdot \underline{e_{1}}= 1 = \underline{e_{2}} \cdot \underline{e_{2}} = \underline{e_{3}} \cdot \underline{e_{3}}$$
We can abbreviate the definition using the {\bf Kronecker-Delta}

\begin{definition}[Kronecker-Delta]

	Let $a,b \in {1,2,3}$. Then we can write:

	\begin{equation}
		\label{eq:kronecker-delta}
		\underline{e_{a}}\cdot \underline{e_{b}}
		=
		\begin{cases}
			1 \ \ \text{if } a = b = 1,2,3 \\
			0  \ \ \text{if } a \neq b
		\end{cases}
		= \delta_{ab}
	\end{equation}

	This will also be useful for calculating scalar product
\end{definition}

\subsection{Scalar product using Kronecker-Delta}
\begin{theorem}[Scalar Product using Kroncker-Delta]
	Let $\vec{a} \vec{b} \in \mathbb{E}^3$. Then
	\begin{equation}
		\vec{a} \cdot \vec{v} = \sum_{i=1}^{3}a_kb_k
	\end{equation}
\end{theorem}
\begin{proof}
	\begin{align*}
		\underline{a} \ \cdot \ \underline{b} & = \Big( \sum\limits_{k = 1}^{3}a_k\underline{e_{k}} \Big)\ \cdot \ \Big( \sum\limits_{l = 1}^{3}b_l\underline{e_{l}} \Big) \\ \\
		                                      & = \sum_{k, \ l} a_{k} \ b_{l} \ \underline{e_{k}}\ \cdot \ \underline{e_{l}}                                               \\ \\
		                                      & = \sum_{k, \ l} a_{k} \ b_{l} \ \delta_{kl}
	\end{align*}
	Now by the definition of Kronecker-Delta \ref{eq:kronecker-delta}, it is 0 for all cases except when $k = l$. where it has a value of {\bf 1} So the summation becomes:
	$$\underline{a} \ \cdot \underline{b} =\sum_{k=1}^{3} a_{k} \ b_{k}$$
\end{proof}
